\documentclass[a4paper,10pt]{article}
\usepackage[utf8]{inputenc}
\usepackage{longtable}
\usepackage{amsmath,amsthm,amssymb}
\usepackage{amsfonts}
\usepackage{enumitem}
\usepackage{color}
%opening
\title{}
\author{}

\begin{document}

\maketitle

\begin{enumerate}
 \item Defina con ejemplos simples la tautología y contradicción.
 \\ Una tautología es una preposición que siempre es cierta, por ejemplo: "$1+1=2$" ya que sin importar quien lo diga, o el contexto siempre es verdad.
 \\ Una contradicción, en cambio siempre es falsa, por ejemplo: "los seres humanos pueden respirar naturalmente bajo el agua", lo cual es mentira.
 
 \item Para cada una de las preposiciones compuestas construya la tabla de verdad.
\begin{enumerate}[label=\alph*]
 \item $(p\to q)\lor(\neg p\to q)$
 \\\begin{tabular}{@{ }c@{ }@{ }c | c@{ }@{}c@{}@{ }c@{ }@{ }c@{ }@{ }c@{ }@{}c@{}@{ }c@{ }@{}c@{}@{ }c@{ }@{ }c@{ }@{ }c@{ }@{ }c@{ }@{}c@{}@{ }c}
p & q &  & ( & p & $\rightarrow$ & q & ) & $\lor$ & ( & $\neg$ & p & $\rightarrow$ & q & ) & \\
\hline 
V & V &  &  & V & V & V &  & \textcolor{red}{V} &  & F & V & V & V &  & \\
V & F &  &  & V & F & F &  & \textcolor{red}{V} &  & F & V & V & F &  & \\
F & V &  &  & F & V & V &  & \textcolor{red}{V} &  & V & F & V & V &  & \\
F & F &  &  & F & V & F &  & \textcolor{red}{V} &  & V & F & F & F &  & \\
\end{tabular}

\item $(p\to q)\land(\neg p\to q)$
\\\begin{tabular}{@{ }c@{ }@{ }c | c@{ }@{}c@{}@{ }c@{ }@{ }c@{ }@{ }c@{ }@{}c@{}@{ }c@{ }@{}c@{}@{ }c@{ }@{ }c@{ }@{ }c@{ }@{ }c@{ }@{}c@{}@{ }c}
p & q &  & ( & p & $\rightarrow$ & q & ) & $\land$ & ( & $\neg$ & p & $\rightarrow$ & q & ) & \\
\hline 
V & V &  &  & V & V & V &  & \textcolor{red}{V} &  & F & V & V & V &  & \\
V & F &  &  & V & F & F &  & \textcolor{red}{F} &  & F & V & V & F &  & \\
F & V &  &  & F & V & V &  & \textcolor{red}{V} &  & V & F & V & V &  & \\
F & F &  &  & F & V & F &  & \textcolor{red}{F} &  & V & F & F & F &  & \\
\end{tabular}
\end{enumerate}

\item Demuestre que si $n$ es un entero positivo cualquiera, entonces.
\\$\dfrac{1}{3}(n^3 + 2n)$
\begin{proof}
 Usando el método de demostracióm por inducción, suponemos que $\dfrac{1}{3}(n^3 + 2n)$ es un número entero. Despues evaluamos.
 \begin{align*}
 L(1) &= \dfrac{1}{3}(0^3 + 2(0))
 \\&= 0
 \\L(2)&=\dfrac{1}{3}(2^3 + 2(2))
 \\&=4
 \\L(k)&=\dfrac{1}{3}(k^3 + 2k) \tag*{Suposición}
 \\L(k+1)&=\dfrac{1}{3}[(k+1)^3 + 2(k+1)]
 \\&=\dfrac{1}{3}(k^3+3k^3+5k+3) \tag*{Desarrollando el cubo}
 \\&=\dfrac{1}{3}(k^3+2k) + \dfrac{1}{3}(3k^2+3k+3) \tag*{Agrupando}
 \\&=\dfrac{1}{3}(k^3+2k) + (k^2+k+1) \qedhere
 \end{align*}
\end{proof}

\item Para el siguiente enunciado en los que $A,B,C$ y $D$ son conjuntos arbitrarios, compruebe que es verdadero o dé un contrajemplo para establecer que es falso.
\\$A\cup C=B\cup C\to A=B$
\\Contra ejemplo:
\\ Sean $A=\{1\}$, $B=\{1,2\}$ y $C=\{2\}$
\\ $A\cup C=\{1,2\}$
\\ $B\cup C=\{1,2\}$
\\pero $A\neq B$

\item Para el siguiente enunciado en los que $A,B,C$ y $D$ son conjuntos arbitrarios, compruebe que es verdadero o dé un contrajemplo para establecer que es falso.
\\$A-(B\times C)=(A-B)\times(A-C)$
\\Contra ejemplo:
\\Sean $A=\{1\}$, $B=\{2\}$ y $C=\{3\}$
\\ $A-(B\times C) = \{1\} - \{(2,3)\} = \{1\}$
\\$(A-B)\times (A-C)=(\{1\}-\{2\})\times(\{1\}-\{3\})=\{(1,1)\}$

\end{enumerate}
\end{document}
