\documentclass[a4paper,10pt]{article}
\usepackage[utf8]{inputenc}

%opening
\title{Auto reflexiones. Unidad 1}
\author{Cristopher Aldama Pérez}

\begin{document}
\maketitle

La unidad 1 tomaba el tema de las preposiciones y reglas de inferencia, es interesante descrubrir que sin saberlo éstas reglas están implicitas en nuestro lenguaje de uso común, muchas veces al resolver los ejercicios de esta unidad, me veía repitiendo lentamente las preposiciones para mí hasta entender su sentido lógico. Hacía esto incluso con reglas no tan intuitivas como los modus, y las equivalencias entre diferentes preposiciones.
\\
\\Por otra parte me fue inesperado descubrir con que facilidad se pueden abstraer sentencias mediante simbolos, para despues manipularlos mediante una serie de reglas bien definidas. Así caigo en cuenta que las matemáticas pueden ser usadas para modelar cosas como el lenguaje y los pensamientos. Anteriormente pensaba equivocadamente que las matemáticas solo trataban con números y problemas relacionados a ellos.
\\
\\Aunque la realización de tablas de verdad me pareció una actividad muy laboriosa, entiendo su utilidad para encontrar contradicciones o certezas en discursos e ideas, mismas que pueden ayudar a evidenciar noticias falsas, malos argumentos retóricos o mentiras.
\\
\\Así pues mi conclusión es que la lógica proposicional es una herramienta aún vigente que nos enseña a pensar de una forma mas ordenada y a descubrir la verdad detrás del lenguaje humano.

\end{document}
