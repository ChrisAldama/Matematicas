\documentclass[a4paper,10pt]{article}
\usepackage[utf8]{inputenc}
\usepackage{amsmath,amsthm,amssymb}
\usepackage{amsfonts}


%opening
\title{Actividad 3 Unidad 3}
\author{Cristopher Aldama Pérez}

\begin{document}

\maketitle

\section{Demostraciones por medio de conjuntos}

\begin{enumerate}
 \item Para cualesquiera dos conjuntos $A$ y $B$, demuestra que:
 \\ $A \subseteq B \iff A - B = \emptyset $
 \begin{proof}
  \begin{align*}
  A \subseteq B &\to
  \\ x \in A \land x \in B &\to
  \\A - B &= \emptyset \tag*{Ya que no todos los elementos A estan en B}
  \\A \subseteq B &\to A - B = \emptyset \tag{1}
  \\A - B = \emptyset&\to \tag*{Por otro lado}
  \\x \in A \land x \notin B = \emptyset&\to
  \\ x \in A \land x \in B&\to     \tag*{Ya que la resta $A - B = \emptyset$}
  \\ A \subseteq B
  A - B = \emptyset &\to A \subseteq B \tag{2}
  \\(A \subseteq B \to A - B = \emptyset) &\land (A - B = \emptyset \to A \subseteq B) \tag*{de 1 y 2}
  \\A \subseteq B &\iff A - B = \emptyset \qedhere
  \end{align*}
 \end{proof}

 \item Para cualesquiera 2 conjuntos $A$ y $B$ demostrar que:
 \\ $(A \cup B)^c = A^c \cap B^c$
\begin{proof}
  \begin{align*}
  (A \cup B)^c &=
  \\ \neg[(x \in A) \lor (x \in B)] &=
  \\ \neg(x \in A) \land \neg(x\in B) &=
  \\ x \notin A \land x \notin B &=
     A^c \cap B^c \qedhere
  \end{align*}
 \end{proof}
 $(A \cap B)^c = A^c \cup B^c $
\begin{proof}
  \begin{align*}
  (A \cap B)^c &=
  \\ \neg[(x \in A) \land (x \in B)] &=
  \\ \neg(x \in A) \lor \neg(x\in B) &=
  \\ x \notin A \lor x \notin B &=
     A^c \cup B^c \qedhere
  \end{align*}
 \end{proof}
 
 \item Para cualesquiera 3 conjuntos $A$, $B$ y $C$, demuestra que
 \\ $A \cup ( B \cap C) = (A\cup B)\cap(A\cup C) $
 \begin{proof}
  \begin{align*}
    A\cup(B\cap C) &=
    \\x\in A \lor(x\in B\land x\in C) &=
    \\(x\in A\lor x\in B)\land(x\in A\lor x\in C) &= 
    (A\cup B)\cap(A\cup C) \qedhere
  \end{align*}
 \end{proof}
 $A \cap ( B \cup C) = (A\cap B)\cup(A\cap C) $
 \begin{proof}
  \begin{align*}
    A\cap(B\cup C) &=
    \\x\in A \land(x\in B\lor x\in C) &=
    \\(x\in A\land x\in B)\lor(x\in A\land x\in C) &= 
    (A\cap B)\cup(A\cap C) \qedhere
  \end{align*}
 \end{proof}
 \item Para cualesquiera conjuntos $A$ y $B$, 
 \\ Si $A \subseteq B \to p(A) \subseteq p(B) $
 \begin{proof}
  \begin{align*}
   p(A) = \{x| x \subset A\}
   \\x\in p(A) \tag*{Sea x un elemento cualquiera de $p(A)$}
   \\x \subseteq A \tag*{Definicion de $p(A)$}
   \\x \subseteq B \tag*{Ya que $A\subseteq B$}
   \\x \in p(B) \tag*{Definicion de p(B)}
   \\ p(A)\subseteq p(B) &&\qedhere
  \end{align*}
 \end{proof}
\end{enumerate}
\end{document}
