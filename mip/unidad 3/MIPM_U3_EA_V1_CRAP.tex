\documentclass[a4paper,10pt]{article}
\usepackage[utf8]{inputenc}
\usepackage{amsmath,amsthm,amssymb}
\usepackage{amsfonts}
\usepackage{enumitem}

%opening
\title{Actividad 3 Unidad 3}
\author{Cristopher Aldama Pérez}

\begin{document}

\maketitle

\section{Demostraciones por medio de conjuntos}

\begin{enumerate}
 \item Contesta lo que se te pide:
 \\En una encuesta realizada a 125 personas respecto a su género de película favorito se obtuvierton los siguintes datos: a 40 les gusta el drama, a 47 las de comedia y a 56 las de acción.
 \\Adicionalmete, se sabe que a 18 les gusta las de drama y comedia, a 20 las de comedia y acción, a 22 drama y acción y finalmete a 12 les gustan los tres géneros. Sean:
   \\ $A = \{$les gusta el Drama$\}$ y $|A| = 40$
   \\ $B = \{$les gusta las de comedia$\}$ y $|B| = 47$
   \\ $C = \{$les gusta las de acción$\}$ y $|C| = 56$
    
  \begin{enumerate}[label=\alph*]
   \item ¿A cuántas personas les gusta un solo género
   \\ Contamos la unión de todas las personas que les gusta algún género y restamos a las que les gusta más de uno.
   \\Sabiendo que: $|A\cup B\cup C| = |A|+|B|+|C|-|A\cap B|-|A\cap C|-|B\cap C| + |A\cap B\cap C|$
   \\ Podemos calcular las personas que solo gustan de un sólo género como:
   \\ $|A|+|B|+|C|-2|A\cap B|-2|A\cap C|-2|B\cap C| + 2|A\cap B\cap C| =$
   \\ $40 + 47 + 56 - 36 - 40 - 44 + 24 = 47$
   \item ¿A cuántas personas no les gusta ninguno de los géneros encuestados?.
   \\ $|U|-|A\cup B\cup C| =$
   \\ $125-(40+47+56-18-20-22+12)=30$
  \end{enumerate}
\item Utiliza las leyes de álgebra de conjuntos para demostrar lo que se pide, justifica tu respuesta.
\begin{enumerate}[label=\alph*]
 \item $(A\cup B)\cap(A\cup\emptyset)=A$
 \begin{proof}
  \begin{align*}
 (A\cup B)\cap(A\cup\emptyset)&=
 \\(A\cup B)\cap A&=
 \\(A\cap A)\cup(A\cap B)&=
 \\A\cup(A\cap B) &= A \iff (A\cap B)=\emptyset
  \end{align*}
 \end{proof}
 
 \item $A-(A\cap B)=A-B$
 \begin{proof}
  \begin{align*}
  A-(A\cap B)&=
  \\A-A\cup A-B&=
  \\\emptyset\cup A-B&=A-B
  \end{align*}
 \end{proof}
 
 \item $(A-B)\cap(B-A)=\emptyset$
 \begin{proof}
  \begin{align*}
  (A-B)\cap(B-A)&=
  \\(A\cap B^c)\cap(B\cap A^c) &=
  \\(A\cap A^c)\cap(B\cap B^c) &= \emptyset
  \end{align*}
 \end{proof}
\item $A-(B\cup C)=(A-B)\cap(A-C)$
 \begin{proof}
  \begin{align*}
    A-(B\cup C)&=
    \\A\cap(B\cup C)^c&=
    \\A\cap B^c\cap C^c&=
    \\(A\cap B^c)\cap(A\cap C^c) &= (A-B)\cap(A-C)
  \end{align*}
 \end{proof}
\end{enumerate}
\item Demuestre: $B-A=B\cap A^c$, Así la diferencia se escribe en términos de las operaciones de intersección y complemento.
\begin{proof}
 \begin{align*}
 B-A &=
 \\\{x|x\in B\land x\notin A\} &= 
 \\\{x|x\in B\}\land\{x|\notin A\}&=B\cap A^c \qedhere
 \end{align*} 
\end{proof}

\item si $A, B$ y $C$ son conjuntos, demuestre tanto analítica como gráficamente que 
\\$A\cap(B-C)=(A\cap B)-(A\cap C)$
\begin{proof}
 \begin{align*}
 A\cap(B-C)&=
 \\A\cap B\cap C^c&= \tag*{Equivalencia de diferencia}
 \\(A\cap B\cap C^c)\cup(A\cap A^c\cap B) &= \tag*{Sabiendo que $A\cap A^c = \emptyset$ y $A\cup\emptyset = A$}
 \\(A\cap B)\cap(A^c\cup C^c)&= \tag*{Refactorizando $A\cap B$}
 \\(A\cap B)\cap(A\cap C)^c&= \tag*{Ley de Morgan}
 \\A(\cap B)-(A\cap C) &&\qedhere
 \end{align*} 
\end{proof}
\end{enumerate}
\end{document}
