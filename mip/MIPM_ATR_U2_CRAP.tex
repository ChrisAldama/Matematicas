\documentclass[a4paper,10pt]{article}
\usepackage[utf8]{inputenc}

%opening
\title{Auto reflexiones. Unidad 2}
\author{Cristopher Aldama Pérez}

\begin{document}
\maketitle
La unidad 2 retoma el temas de las demostraciones matemáticas, las cuales considero son un tema muy interesante, ya que aveces estas parecen ser como rompe cabezas y laberintos que aunque dificiles, tambien son muy gratificantes al ser resueltas. 
\\
\\En particular las demostraciones por inferencia me costaron un poco más de trabajo ya que no estaba acostumbrado a pensar de esa manera. Pero al ir resolviendo ejercicios entiendo ahora por qué funcionan y como se puede hacer uso de la inferencia, para encontrar soliciones.
\\
\\Respondiendo a la pregunta, si considero a las demostraciones matemáticas útiles en la vida diaria, diría que si, ya que como ingeniero de software estoy acostumbrado a lidiar con problemas que aunque pueden resolverse de varias maneras, aveces hay una que se acompla mejor al contexto o limitaciones de dicho problema. Así pues las demostraciones nos muestran que hay diversos caminos y diferentes formas de pensar para encontrar soluciones.
\\
\\En conlución, los métodos de demostración pueden ser útiles y apliaciones en la vida, además de mejorar y ejercitar nuestra manera de pensar.
\end{document}
