\documentclass[a4paper,10pt]{article}
\usepackage[utf8]{inputenc}

%opening
\title{Activida 2. Conectivos lógicos}
\author{Cristopher Aldama Pérez}

\begin{document}

\maketitle

\section{Conectivos lógicos}
Instrucciones: Identifica si es una proposición simple o compuesta, cuando suceda el segundo caso separa las preposiciones y escribe el conectivo lógico que los une.

\begin{enumerate}
\item Si la suma de los ángulos interiores de un polígono vale dos rectos el polígono es un triángulo.\\
  Esta preposición es compleja y está formada por dos preposiciones: ``La suma de los ángulos intetiores de un polígono vale dos rectos'', ``el polígono es un triángulo'' y están unidas por el conector lógico \(\to\) de implicación.

\item Si una recta tiene dos puntos comunes con un plano, toda la recta está contenida 
en el plano.\\
Es un preposición compleja formada por: "Una recta tiene dos puntos comunes con un plano", "toda la recta está contenida en el plano" y unidas por por el conector lógico \(\to\) de implicación.

\item El  dominio  de  una  función  está  formado  por  el  conjunto  de  todos  los  valores posibles de '\(x\)' y el  contradominio de la función está formado por todos los valores posibles de '\(y\)'.\\
Es un preposición compleja formada por "El  dominio  de  una  función  está  formado  por  el  conjunto  de  todos  los  valores posibles de x" y "El  contradominio de la función está formado por todos los valores posibles de y." las cuales están unidas por la \( \land \) disyunción.

\item Una función es trascendente si no puede expresarse mediante un número finito de sumas, diferencias, productos, cocientes y raíces.\\
Es una prepoción compleja donde "Una función que no puede expresarse mediante un número finito de sumas, diferencias, productos, cocientes y raíces" implica \(\to\) a "Una función es trascendente".

\end{enumerate}
 
\end{document}
