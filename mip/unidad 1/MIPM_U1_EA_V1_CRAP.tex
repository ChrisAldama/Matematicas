\documentclass[a4paper,10pt]{article}
\usepackage[utf8]{inputenc}
\usepackage{color}
\input {fitch}

%opening
\title{Evidencia de Aprenduzaje. Unidad 1}
\author{Cristopher Aldama Pérez}

\begin{document}

\maketitle

\begin{enumerate}
 \item De a cuerdo a la siguiente operación con proposiciones blueacte un problema con el cual cumpla con las preposiciones simples p, q, r y verifique la operación siguiente.
 \[[(p \to q) \land (q \to r)] \to (p \to r)\]
 Sean:
 \begin{itemize}
  \item \(p\): Ahorro dinero
  \item \(q\): Vengo dinero
  \item \(r\): Compro un departamento
 \end{itemize}
 De tal forma que "Si ahorro dinero entonces tengo dinero y si tengo dinero puedo comprar un departamento, entonces Si ahorro dinero entonces puedo comprar un departamento", asi pues:
 \[
  \begin{nd}
   \hypo {1} {p \to q}
   \hypo {2} {q \to r}
   \have {3} {\neg p \lor q} \by{EI}{1}
   \have {4} {\neg p \lor p}
   \have {5} {\neg p \land (p \lor q)} \by{LD}{4, 1}
   \have {6} {\neg p} \by{LCI}{5}
   \have {7} {\neg r \lor r}
   \have {8} {r \land (\neg q \lor \neg r)} \by{LD}{7, 2}
   \have {9} {r} \by{LCI}{8}
   \have {10} {\neg p \lor r} \by{ID}{6, 9}
   \have {11} {p \to r} \by{EI}{10}
  \end{nd}
 \]

\item Sean \(p, q, r\) tres preposiciones simples dadas como sigue:
\begin{itemize}
 \item \(p\): Iván estudia.
 \item \(q\): Iván juega fútbol.
 \item \(r\): Iván aprueba el semestre
\end{itemize}
Identifica las premisas y su conclusión y demuestra por medio de tabla de verdad si es tautología o contradicción.

 \[[(p \to q) \land (q \to r)] \to (p \to r)\]
 
\begin{tabular}{@{ }c@{ }@{ }c@{ }@{ }c | c@{ }@{}c@{}@{}c@{}@{ }c@{ }@{ }c@{ }@{ }c@{ }@{}c@{}@{ }c@{ }@{}c@{}@{ }c@{ }@{ }c@{ }@{ }c@{ }@{}c@{}@{}c@{}@{ }c@{ }@{}c@{}@{ }c@{ }@{ }c@{ }@{ }c@{ }@{}c@{}@{ }c}
p & q & r &  & ( & ( & p & $\rightarrow$ & q & ) & $\&$ & ( & q & $\rightarrow$ & r & ) & ) & $\rightarrow$ & ( & p & $\rightarrow$ & r & ) & \\
\hline 
V & V & V &  &  &  & V & V & V &  & V &  & V & V & V &  &  & \textcolor{blue}{V} &  & V & V & V &  & \\
V & V & F &  &  &  & V & V & V &  & F &  & V & F & F &  &  & \textcolor{blue}{V} &  & V & F & F &  & \\
V & F & V &  &  &  & V & F & F &  & F &  & F & V & V &  &  & \textcolor{blue}{V} &  & V & V & V &  & \\
V & F & F &  &  &  & V & F & F &  & F &  & F & V & F &  &  & \textcolor{blue}{V} &  & V & F & F &  & \\
F & V & V &  &  &  & F & V & V &  & V &  & V & V & V &  &  & \textcolor{blue}{V} &  & F & V & V &  & \\
F & V & F &  &  &  & F & V & V &  & F &  & V & F & F &  &  & \textcolor{blue}{V} &  & F & V & F &  & \\
F & F & V &  &  &  & F & V & F &  & V &  & F & V & V &  &  & \textcolor{blue}{V} &  & F & V & V &  & \\
F & F & F &  &  &  & F & V & F &  & V &  & F & V & F &  &  & \textcolor{blue}{V} &  & F & V & F &  & \\
\end{tabular}

\item Si Julieta se casa, es porque consiguió el préstamo, Julieta consiguió el préstamo, por lo tanto Julieta se casará.\\
Demuestre si es tautología o contradicción por medio de tabla de verdad.
\begin{itemize}
 \item \(p\): Julieta de casa.
 \item \(q\): Julieta consiguió el préstamo.
\end{itemize}

\[(p \to q) \land (q \to p)\]

\begin{tabular}{@{ }c@{ }@{ }c | c@{ }@{}c@{}@{ }c@{ }@{ }c@{ }@{ }c@{ }@{}c@{}@{ }c@{ }@{}c@{}@{ }c@{ }@{ }c@{ }@{ }c@{ }@{}c@{}@{ }c}
p & q &  & ( & p & $\rightarrow$ & q & ) & $\&$ & ( & q & $\rightarrow$ & p & ) & \\
\hline 
V & V &  &  & V & V & V &  & \textcolor{blue}{V} &  & V & V & V &  & \\
V & F &  &  & V & F & F &  & \textcolor{red}{F} &  & F & V & V &  & \\
F & V &  &  & F & V & V &  & \textcolor{red}{F} &  & V & F & F &  & \\
F & F &  &  & F & V & F &  & \textcolor{blue}{V} &  & F & V & F &  & \\
\end{tabular}

\end{enumerate}
\end{document}
