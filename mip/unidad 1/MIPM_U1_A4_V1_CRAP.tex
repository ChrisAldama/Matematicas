\documentclass[a4paper,10pt]{article}
\usepackage[utf8]{inputenc}
\input{fitch}

%opening
\title{Actividad 4. Reglas de inferencia y premisas}
\author{Cristopher Aldama Pérez}

\begin{document}

\maketitle

\section{Reglas de Inferencia}
Instrucciones: Utiliza las reglas de inferencia para demostrar la conclusión que se pide, justifica tu respuesta.

\begin{enumerate}
 \item En la escuela Juan tuvo 10 de promedio en matemáticas o 10 de promedio en física, si Juan tuvo 10 de promedio en matemáticas entonces se ganará un viaje y si Juan tuvo 10 de promedio en física entonces se ganará un viaje, lo anterior es suficiente para que Juan se ganará un viaje.\\
 Pasando las preposiciones a lenguaje formal:
 \begin{itemize}
  \item \(p\): Juan tuvo 10 de promedio en matemáticas
  \item \(q\): Juan tuvo 10 de prmiedio en física.
  \item \(r\): Juan se ganará un viaje.
 \end{itemize}
Luego usando las reglas de inferencia tenemos que probar que \(r\):
\[
\begin{nd}
  \hypo {1} {p \lor q}
  \hypo {2} {p \to r}
  \hypo {3} {q \to r}
  \have {4} {r \lor r} \by{DC}{1,2,3}
  \have {5} {r} \by{LSD}{4}
\end{nd}
\]

\item Si vendo muchas mercancía, entonces tendré mucho dinero y no es cierto que tengo mucho dinero, quiere decir que no vendo mucha mercancía.\\
Pasando las preposiciones a lenguaje formal:
\begin{itemize}
 \item \(p\): Vendo mucha mercancía.
 \item \(q\): Tengo mucho dinero.
\end{itemize}
Luago usando las reglas de inferencia tenemos que probar que \(\neg p\).
\[
 \begin{nd}
   \hypo {1} {p \to q}
   \hypo {2} {\neg q}
   \have {3} {\neg p} \by{MTT}{1, 2}
 \end{nd}
\]

\item Si ganas la olimpiada tus compañeros de escuela se ponen alegres, y si tus compañeros están alegres tus adversarios se ponen tristes, nos encontramos, en consecuencia que si tú ganas la olimpiada. tus adversarios se ponen tristes.\\
Pasando las preposiciones a lenguaje formal:
\begin{itemize}
 \item \(p\): Ganas la olimpiada.
 \item \(q\): Tus compañeros de la escuela están alegres.
 \item \(r\): Tus adversarios están trsites.
\end{itemize}
Luego usando las reglas de inferencia tenemos que probar que \(p \to r\).
\[
 \begin{nd}
   \hypo {1} {p \to q}
   \hypo {2} {q \to r}
   \have {3} {p \to r} \by{SH}{1,2}
 \end{nd}
\]
\item Si invierto más dinero en un negocio, entonces aumentarán mis ganancias, como es cierto que invito más dinero, concluyo que aumentarán mis ganancias.\\
Pasando las preposiciones a lenguaje formal:
\begin{itemize}
 \item \(p\): Invierto más dinero.
 \item \(q\): Aumentan mis ganancias.
\end{itemize}
Luego usando las reglas de inferencia tenemos que probar que \(q\).
\[
 \begin{nd}
  \hypo {1} {p \to q}
  \hypo {2} {p}
  \have {3} {q} \by{MPP}{1, 3}
 \end{nd}
\]

\end{enumerate} 




\end{document}
