\documentclass[a4paper,10pt]{article}
\usepackage[utf8]{inputenc}

%opening
\title{Auto reflexiones. Unidad 3}
\author{Cristopher Aldama Pérez}

\begin{document}
\maketitle

La unidad 3 nos hablas sobre la teoría de conjuntos, y como estos nos ayudan a crear colecciones de cosas mediante reglas o expresiones. Y como los conjuntos en sí, pueden ser manipulados para moldear problemas y sus soluciones, además de hacer uso de la lógica para definir sus propiedades.
\\
\\Al inicio encontré un poco complidado entender como funcionan las demostraciones y propiedades de los conjuntos, pero al darme cuenta que su notación está basada en la lógica proposicional, me pareció mas simple, dándome cuenta que el truco está en demostrar que cierto elemento está o no, incluido en cierto conjunto.
\\
\\Al final descubro que me gusta mucho el tema y me gustaría seguir indagando al respecto.
\end{document}
