\documentclass[a4paper,10pt]{article}
\usepackage[utf8]{inputenc}
\usepackage{amsmath,amsthm,amssymb}
\usepackage{amsfonts}

%opening
\title{Evidencia de aprendizaje Unidad 2}
\author{Cristopher Aldama Pérez}

\begin{document}

\maketitle

\section{Demuestra los enunciados por medio de los diferentes métodos de demostración que consideres adecuado}

\begin{enumerate}
 \item Demostrar que no hay ningún número racional cuyo cuadrado sea 15.
   \begin{proof}
    Usando el método de reducción al absurdo, negamos la preposición y suponemos que $\sqrt{15}$ es racional, lo que implica que:
      \begin{align*}
        \sqrt{15} &= \dfrac{p}{q}
        \\ 15 &= \dfrac{p^2}{q^2}
        \\ 15q^2 &= p^2
        \\ \tag*{Así $p$ es un multiplo de 15 por lo que puede ser escrito como: $p=15k$}
        \\ 15q^2 &= (15k)^2
        \\ q^2 &= 15k^2
      \end{align*}
      Así $q$ tambien es multiplo de 15, lo cual significa que tanto $p$ como $q$ tienen 15 como factor común, lo cual contradice nuestra suposición inicial
   \end{proof}
   
  \item Si $x$ es racional y distinto de cero y $y$ es irracional, entonces $(x + y)$ y $(x y)$ son racionales.
  \begin {proof}
  Si $x$ es racional se cumple que $x = \dfrac{a}{b}$, en cambio $y = c$ ya que es irracional, así:
    \begin{align*}
     x + y &= \dfrac{a}{b} + c
     \\ &= \dfrac{a}{b} + \dfrac{cb}{b} \tag*{Multiplando $c$ por $\dfrac{b}{b}$}
     \\ &= \dfrac{a + cb}{b}            \tag*{Introduciendo $d = a + cb$}
     \\ &= \dfrac{d}{b}
     \\ xy &= \dfrac{a}{b} c            \tag*{Introduciendo $e = ac$}
     \\ &= \dfrac{e}{b}                 &&\qedhere
    \end{align*}
  \end {proof}
  
  \item Sea $x \in \mathbb{R}$, demuestre que si $|x + y| > |x| + |y|$ entonces $y$ no es un número real.
  \begin{proof}
   Sabiendo que $p \to q \equiv \neg q \to \neg p$ podemos reformular la premisa como: si $y$ es un número real entonces $|x + y| <= |x| + |y|$ ya que $x$ y $y$ son números reales podemos definir el valor absoluto como: $|a| = \sqrt{a^2}$.
   \begin{align*}
   |x + y| &= \sqrt{(x + y)^2}
   \\ &= x + y
   \\ &= \sqrt{x^2} + \sqrt{y^2}
   \\ &= |x| + |y| &&\qedhere
   \end{align*}
  \end{proof}

  \item Pruebe por inducción que $ 5+9+13+...+(4n+1)=n(2n+3) $.
  \begin{proof}
   \begin{align*}
   L(1) &= \sum_{1}^{1} 4n+1 
   \\ &= 4(1) + 5
   \\ &= (1)[2(1) + 3]
   \\ &= \sum_{1}^{1} n(2n + 3)
   \\ L(2) &= \sum_{1}^{2} 4n + 1 
   \\ &= 4(1) + 5 + 4(2) + 5
   \\ &= (1)[2(1) + 3] + (2)[2(2) + 3]
   \\ &= \sum_{1}^{2} n(2n + 3)
   \\ L(k) &= \sum_{1}^{k} 4n + 1
   \\ &= \sum_{1}^{k} n(2n + 3)         \tag*{suposición}
   \\ L(k + 1) &= \sum_{1}^{k + 1} 4n + 1
   \\ &= 4(k + 1) + 1 + \sum_{1}^{k} 4n + 1
   \\ &= 4(k + 1) + 1 + k(2k + 3)
   \\ &= 2k^2 + 7k + 5
   \\ &= (k + 1)(2k + 5)
   \\ &= (k + 1)[2(k + 1) + 3] &&\qedhere
   \end{align*}
  \end{proof}

  \item Probar por inducción que $1+3+5+...+(2n-1)=n^2$
  \begin{proof}
   \begin{align*}
    L(1) &= \sum_{1}^{1} 2n - 1
    \\ &= 2(1) - 1
    \\ &= 1^2
    \\ L(2) &= \sum_{1}^{2} 2n - 1
    \\ &= 2(1) - 1 + 2(2) - 1
    \\ &= 2^2
    \\ L(k) &= \sum_{1}^{k} 2n - 1
    \\ &= k^2
    \\ L(k+1) &= \sum_{1}^{k + 1} 2n - 1
    \\ &= 2(k + 1) - 1 + \sum_{1}^{k} 2n - 1
    \\ &= k^2 + 2k + 1 \tag*{inducción}
    \\ &= (k + 1)^2 &&\qedhere
   \end{align*}

  \end{proof}


\end{enumerate}
\end{document}
